\documentclass{article} 
\usepackage{color} 
\usepackage{forest}
\usepackage{mathtools}
\usepackage[utf8]{inputenc}
\DeclarePairedDelimiter{\ceil}{\lceil}{\rceil}
\begin{document} 
\title{TDT4225 -- Assignment 4} 
\author{Filip F Egge}
\date{November 12, 2015} 
\maketitle

\newpage
\subsection*{Algebra with filter}
Joining of two tables $A$ and $B$, $V_A = 400\:B \times 1\:000\:000 = 4 \times 10^8 = 400\:MB$, $V_B = 600\:B \times 10\:000\:000 = 6 \times 10^9 = 6000\:MB$,
$V_R = 600\:B \times 1\:000\:000 = 6 \times 10^8 = 600\:MB$. Our workspace is $20\:MB$, this gives $n = \ceil{\frac{V_A}{M}} = \ceil{\frac{400\:MB}{20\:MB}} = 20$.
$V_{nl}^J = V_A + nV_B + V_R = 400\:MB + 20 \times 6000\:MB + 600\:MB = 121\:GB$

Join with filter.

\[
    V = 2V_A - M + V_B + V_B \delta_{F_A} \left( \ceil*{\frac{(V_A - M) \delta_{F_{A\cap B}}}{M}}\right) + V_R
\]
I did not manage to calculate this join with filter.

\subsection*{Parallel algebra}
\subsubsection*{Describe the different partitioning methods used in parallel algebra.}
Horizontal fragmentation divides the table into fragment, each fragment corresponds to a group of records stored on a node.
Vertical fragmentation has records with key and one or a few attributes for each fragment. The third option is a 
mix between the two others. 
\subsubsection*{Why is hashing a very good method?}
Using hashing is a good way to place records, this requires to indexes and a good hash formula.
\subsection*{Dynamo}
Explain the following concepts/techniques used in Dynamo
\begin{description}
    \item[consistent hashing] \hfill \\
        Consistent hashing uses a hashing function whose output is represented as a fixed circular space. Each node is assigned a
        space on this ring, and each item is assigned to a node by hashing the items key. Each node is thus responsible for
        keeping track of the region between itself and the previous node. 
    \item[vector clocks] \hfill \\
        Dynamo uses vector clocks in order to capture causality between different versions of the same object.
    \item[sloppy quorum and hinted handoff] \hfill \\
        Sloppy quorum is a less strict version of the traditional quorum. All read and write operations are performed on the first N healthy
        nodes from the preference list. \\
        Hinted handoff ensures that read and write operations are not failed due to temporary node or network failures.
    \item[merkle trees] \hfill \\
        Dynamo uses merkle trees to detect inconsistency between replicas. A merkle tree is a hash tree where leaves are hashes of the values
        of individual keys.
    \item[gosip-based membership protocol] \hfill \\
        Dynamo uses a completely decentralized membership protocol, and updates are spread using gosip or word of mouth.
\end{description}
\subsection*{RamCloud}
\subsubsection*{How does RamClouds ensure "durability" of data?}
RamClouds can use different ways to ensure durability of data, the most common is to "buffered logging" which uses both disk and memory 
for backup. A local copy is stored in the primary server DRAM and copies stored in two or more backup servers. On write the primary send a 
log entry to the backup server who updates its copy.
\subsubsection*{How does Ousterhout argue that RAMCloud's potential to support ACID transactions is better than for traditional disk-based distributed databases?}
Ousterhout argues that RamClouds low latency and fast transactions limits the use for ACID. Since ensuring ACID scales poorly and adds time
delays to transactions, a trend in storage systems is to give up some ACID properties to improve scalability.
\subsection*{Facebook TAO}
\subsubsection*{How does Facebook TAO solve the problem that the social graph spans the whole world, and that the data should be close to the user?}
Data is divided into logical \textit{shards}, each of which is stored in a logical database. Database servers are responsible for one or more shards. This enables Facebook to place shards at physical servers close to the places the shards relate to.

\subsection*{Google Spanner}
\subsubsection*{How are TimeStamps used in Spanner's transactions?}
Timestamps in Google Spanner is used in versioning, where each version of data is timestamped at commit time. Transactional reads and writes
use two-phase locking. Transaction can be assigned timestamps when all locks have been acquired, but before any has been released.

\end{document}
