\documentclass{article} 
\usepackage{color} 
\usepackage{forest}
\usepackage{mathtools}
\usepackage[utf8]{inputenc}
\DeclarePairedDelimiter{\ceil}{\lceil}{\rceil}
\begin{document} 
\title{TDT4225 -- Assignment 3} 
\author{Filip F Egge}
\date{October 29, 2015} 
\maketitle

\newpage

\subsection*{Explain the adaptive replacement cache of ZFS, with focus on how
it supports different access patterns of blocks?} 
The adaptive replacement cache or ARC for short is a way of managing cache. It
splits the cache into two parts, one holding blocks that have only been
referenced once, and the other holding blocks that have been referenced twice or
more.  ZFS changes the size of the parts dynamically by using four lists of
blocks; $t_1, t_2, b_1, b_2$. $t_1$ containing blocks that have been cached
after being referenced, $t_2$ containing blocks re-referenced while in the first
list. $b_1$ and $b_2$ contain reference to blocks that are evicted from cache,
$b_1$ are those evicted from $t_1$ and $b_2$ those from $t_2$. When a block in
$b_1$ is referenced, the algorithm makes room in $t_1$ at the expense of $t_2$
and vice versa.

\subsection*{Under which conditions may the use of a Bloom filter be
appropriate?} Bloom filter are commonly used to reduce the number of records at
an early state. Since bloom filters will never produce a false negative, only
false positives, it is mostly used to exclude records from a search.

\subsection*{Give an overview of LevelDB's architecture showing how put and
get are using the log, MemTable and SSTables.} 
In LevelDB the log is used to store a sequence of recent updates. When the log
reaches its max size, it will be converted into a sorted table and a new log
file is created. A copy of the log is kept in a memtable. This copy is consulted
on every get so that get operations reflect all logged updates. 

\subsection*{Give an overview of LMDB's architecture showing how put and get
are handled}
LMDB uses a copy-on-write B+ Tree to store data. This means that when LMDB
executes a write operation, it will create a new copy of the B+ tree. This
enables read operations to occur concurrently. When the write operation is
committed, the new tree becomes the "current" tree.

\subsection{SSD vs HDD}
\[
    20\,000 \times 2KB = 40\,000 KB = 40 MB
\]
\[
    40 MB \div 250 MB/S = 0.16 S
\]
\[
    40 MB \div 800KB = 50 tracks/rounds
\]
\[ 
    50 / 0.16 S = 312.5 RPS =  18\,750 RPM
\]

The disk will need to spin at 18750 RPM to be able to boot as fast as the SSD

\subsection*{Sorting 1}
\subsubsection*{How long time is needed by the CPU for initial sort (CPU time
    for pulling the records through
the looser tree)?}

You are to sort a file with 200 million records, record length is 100 byte, and
the key is 12 byte. You have a work space of 200 MB.
Our workspace holds $200 MB / 112 B \approx 1\,785\,714$ records. This gives us a
height of the tournament tree of $h = \lceil log_2 1\,785\,714 \rceil = 21$.
This means we have to perform 17 comparisons for each run. This gives us $17
\times 10^{-6} \times 200 \times 10^6 = 3400\: s = 56$ minutes and $40$ seconds.

\subsubsection*{Assume I/O and CPU to work in parallel. How long time does the
complete sort take?}
Number of runs is $ N = \frac{22\:400\:MB}{2 \times 200\:MB} = 56 $
To calculate I/O time we use the following formula. Where $b$ is block size, in
this case 4 KB. $c$ is transfer time per volume unit, $bc$ is here 0.1
microseconds. $a$ is the access overhead, on SSD this is about 0.1 ms.
\[
    T_V^S = \frac{V}{b}\left(a+bc\right)
\]
\[
    T_V^S = \frac{22\:400\:MB}{4\:KB}(0.1 Ms + 0.1 \mu s) \approx 
    \mbox{9 minutes 20 seconds}
\]
Total sort time is them 1 hour and 4 minutes.

\subsection*{Sorting 2}
\subsubsection*{a)}
Keys in file: 32,44,10,3, 5,79,64,43,98,33,8,3,5,2. 
When sorted using reservoir method with a space of 4 record in each reservoir, they create 3 runs.
They are runs 3-5-10-32-43-44-64-79-98, 3-5-8-33, and 2.
\subsubsection*{b)}
We find the number of dummy runs by using the following formula:
\[
    y = \ceil*{\frac{N-1}{p-1}} = \ceil*{\frac{370-1}{90-1}} = 5
\]
\[
    x = y(p - 1) + 1 - N = 5(90-1) + 1 - 370 = 76
\]
We have to add 76 dummy files do get an optimal merge tree

\subsubsection*{c)}
We calculate the total I/O volume for the whole merge process.
\( V_S = 2V(w+1) = 740(log_{90}370 + 1) = 1712.48 \)


\subsection*{Algebra 1}
\subsubsection*{Nested loop}
B is read $n$ times, \( n = \ceil*{\frac{V_B}{M}} = \ceil*{\frac{300\:00 \times 600\:B}{10\:MB}} = 18 \)
\[
    V_{nl}^J = V_A + nV_B + V_R = 180\:MB + 18\times360\:MB + 50\:MB = 6710\:MB
\]
\subsubsection*{Sort}
\[
    V_{s1}^J = 5V_A + 5V_B + V_R = 5 \times 180\:MB + 5\times 360\:MB + 50\:MB = 2030\:MB
\]

\subsubsection*{Partitioning}
\[
    V_p^J = 3(V_A + V_B) +V_R = 3(180\:MB + 360\:MB) + 50\:MB = 1670\:MB
\]

The best algorithm is Partitioning.


\subsection*{Algebra 2}
Transport volume for a projection. $9\times10^6$ people in Sweden, each represented in a row of $150$ Bytes.
$V_A = 9\times10^6 \times 150\:B = 1350\:MB$. $10^5$ different first names, $V_R = 10^5 \times 30\:B = 3\:MB$.
\\
Plain:
\[
    V_{nl}^P = \frac{V_A(V_A+M)}{2M}+V_R = 
    \frac{1350\:MB \times (1350\:MB + 50\:MB)}{100\:MB}+3\:MB = 18\:903\:MB
\]
\\
Guarding keys:
$\alpha$ - record lenght A, $\kappa$ - key lenght, $\rho$ - record length R
\[
    n = \ceil*{\frac{V_A\kappa}{M\alpha}} = \ceil*{\frac{1350\:MB \times 8\:B}{50\:MB \times 150\:B}} = 2
\]
\[
    V_{nl_w}^P = V_A\left(1+\frac{(n-1)\rho}{\alpha}\right) + V_R = 
    1350\:MB\left(1+\frac{(2-1)30\:B}{150\:B}\right)+3\:MB = 1623\:MB
\]











\end{document}
