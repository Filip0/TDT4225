\documentclass{article}
\usepackage{color}
\usepackage[utf8]{inputenc}
\begin{document}
\title{TDT4225 -- Assignment 2}
\author{Filip F Egge}
\date{October 8, 2015}
\maketitle

\newpage

\subsubsection{Explain the adaptive replacement cache of ZFS, with focus on how it supports different access
patterns of blocks?}
The adaptive replacement cache or ARC for short is a way of managing cache. It splits the cache into two parts, one holding blocks that
have only been referenced once, and the other holding blocks that have been referenced twice or more.
ZFS changes the size of the parts dynamicly by using four lists of blocks; $t_1, t_2, b_1, b_2$. $t_1$ containing blocks that have been
cached after beeing referenced, $t_2$ containing blocks re-referenced while in the first list. $b_1$ and $b_2$ contain reference 
to blocks that are evicted from cache, $b_1$ are those evicted from $t_1$ and $b_2$ those from $t_2$. When a block in $b_1$ is referenced, the algorithm makes room in $t_1$ at the expence of $t_2$ and vice versa.

\subsubsection{Under which conditions may the use of a Bloom filter be appropriate?}
Bloom filter
\end{document}
